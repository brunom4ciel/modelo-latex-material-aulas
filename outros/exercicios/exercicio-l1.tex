%-----PROVA Template
%Adapted by: Milena Lima&Fernanda D.Gomes&Sara R. Pacheco
%Autor:Hirschhorn,Philip(Using the exam document class)
%Adaptations:
%*Tradução da tabela de pontuação;
%*Inclusão de "meios" em forma decimal nos valores pontuais %das questões;
%*Inclusão de Referências e citações.
%Science Project at school 2017-FAPEAM
%View:https://www.overleaf.com/read/bmbsmqkxwndn
%=========================================================
%------------PROVA OBJETIVA/DISCURSIVA
%=========================================================
\documentclass[12pt,answers]{exam}

% usando tema personalizado. 
% arquivo exam-estilo.sty deve estar no mesmo diretório do prova.tex
\usepackage{outros/exercicios/exam-estilo}



%%%%%%%%%%%%%%%%%%%%%%%%%%%%%%%%%%%%
% declaração de nomes de comandos
%
\providecommand{\titulo}[1]{\renewcommand{\titulo}{#1}}
\providecommand{\autor}[1]{\renewcommand{\autor}{#1}}
\providecommand{\data}[1]{\renewcommand{\data}{#1}}
\providecommand{\instituicao}[1]{\renewcommand{\instituicao}{#1}}
\providecommand{\nomecurso}[1]{\renewcommand{\nomecurso}{#1}}
\providecommand{\programa}[1]{\renewcommand{\programa}{#1}}
\providecommand{\emailprograma}[1]{\renewcommand{\emailprograma}{#1}}
\providecommand{\siteprograma}[1]{\renewcommand{\siteprograma}{#1}}
\providecommand{\local}[1]{\renewcommand{\local}{#1}}
\providecommand{\disciplina}[1]{\renewcommand{\disciplina}{#1}}
\providecommand{\turma}[1]{\renewcommand{\turma}{#1}}
\providecommand{\cargahoraria}[1]{\renewcommand{\cargahoraria}{#1}}
\providecommand{\alocacao}[1]{\renewcommand{\alocacao}{#1}}
\providecommand{\figuralogo}[1]{\renewcommand{\figuralogo}{#1}}
\providecommand{\figuralogocapa}[1]{\renewcommand{\figuralogocapa}{#1}}
\providecommand{\figuradmca}[1]{\renewcommand{\figuradmca}{#1}}
% %
% %%%%%%%%%%%%%%%%%%%%%%%%%%%%%%%%%%%%




\makeatletter%
\@ifclassloaded{beamer}%
{
%----------------------------------------------------------
% glossário e acrônimos
%
\usepackage[acronym]{glossaries} %GLOSSÁRIO
%\GlsSetXdyCodePage{utf8}
\glsnoexpandfields
\glsaddall
\makeglossaries
\newglossarystyle{dotglos}{%
	\setglossarystyle{list}%
	\renewcommand*{\glossentry}[2]{%
		\item[\glsentryitem{##1}\glstarget{##1}{\glossentryname{##1}}]
		\ifglshassymbol{##1}{[\glossentrysymbol{##1}]\quad}{}%
		\emph{\glossentrydesc{##1}}%
		\unskip\leaders\hbox to 2.9mm{\hss.}\hfill##2}%
	\renewcommand*{\glsgroupskip}{}%
}
% \newglossarystyle{dotglos}{%
% 	\setglossarystyle{list}% base this style on the list style
% 	\renewcommand*{\glossentry}[2]{%
% 		\item[\glsentryitem{##1}%
% 		\glstarget{##1}{\glossentryname{##1}}]
% 		\glossentrydesc{##1}\glspostdescription\space
% 		\unskip\leaders\hbox to 2.9mm{\hss.}\hfill\ifnum\glsentryprevcount{##1}>1 pp.\else p.\fi\ ##2}%
% }
\setglossarystyle{dotglos}
%----------------------------------------------------------
	}%
{}%
\makeatother%


%----------------------------------------------------------
% alinhar texto justificado
%
\usepackage{ragged2e}
%----------------------------------------------------------



\newcommand\sbullet[1][.5]{\mathbin{\ThisStyle{\vcenter{\hbox{%
					\scalebox{#1}{$\SavedStyle\bullet$}}}}}%
}


% para codigos


\usepackage{listings}
\usepackage{listingsutf8}
\usepackage{inconsolata}

\definecolor{dkgreen}{rgb}{0,0.6,0}
\definecolor{gray}{rgb}{0.5,0.5,0.5}
\definecolor{mauve}{rgb}{0.58,0,0.82}

\lstdefinestyle{CBruno}{
  inputencoding=utf8,
%   extendedchars=false,
    language=java,
    backgroundcolor=\color{white},
    commentstyle=\color{dkgreen},
    % keywordstyle=\color{blue},
    % keywordstyle={[2]\color{magenta}},
    numberstyle=\tiny\color{gray},
    stringstyle=\color{mauve},
    % basicstyle=\footnotesize,
    basicstyle=\tiny,
    comment=[l]{\#},
    escapechar=@,
    escapeinside={\%*}{\text{\%}*)},
    % escapeinside={;@}{\^^M},
    otherkeywords={*,...},
    % escapeinside={\%*}{*)},
    % keywords={@relation,@attribute,@data},
    % morekeywords=[2]{real,integer,numeric,string,date},
    breakatwhitespace=true,
    breaklines=true,
    captionpos=b,
    keepspaces=true,
    firstnumber=1,
    numbers=left,
    numbersep=5pt,
    showspaces=false,
    showstringspaces=false,
    showtabs=false,
    tabsize=2,
    frame=single,
    rulecolor=\color{black},
    keywordstyle=\color{blue},
    % morekeywords={*,...},
    % alsoletter={., [\_]},
    texcl=true,
    alsoletter ={_},
    otherkeywords = {!,!=,~,$,*,\&,\%/\%,\%*\%,\%\%,<-,<<-},
    % morecomment=[s][\color{black}]{<!--}{-->},
    stepnumber=1,
    % morecomment=[l][]{//}, 
    % morecomment=[s][]{/*}{*/},
    % morestring=[b]",
    % morestring=[b]',
    extendedchars=true,
    literate=*
    {á}{{\'a}}1 {é}{{\'e}}1 {í}{{\'i}}1 {ó}{{\'o}}1 {ú}{{\'u}}1
    {Á}{{\'A}}1 {É}{{\'E}}1 {Í}{{\'I}}1 {Ó}{{\'O}}1 {Ú}{{\'U}}1
    {à}{{\`a}}1 {è}{{\`e}}1 {ì}{{\`i}}1 {ò}{{\`o}}1 {ù}{{\`u}}1
    {À}{{\`A}}1 {È}{{\'E}}1 {Ì}{{\`I}}1 {Ò}{{\`O}}1 {Ù}{{\`U}}1
    {ä}{{\"a}}1 {ë}{{\"e}}1 {ï}{{\"i}}1 {ö}{{\"o}}1 {ü}{{\"u}}1
    {Ä}{{\"A}}1 {Ë}{{\"E}}1 {Ï}{{\"I}}1 {Ö}{{\"O}}1 {Ü}{{\"U}}1
    {â}{{\^a}}1 {ê}{{\^e}}1 {î}{{\^i}}1 {ô}{{\^o}}1 {û}{{\^u}}1
    {ã}{{\~a}}1 {ẽ}{{\~e}}1 {ĩ}{{\~i}}1 {õ}{{\~o}}1 {ũ}{{\~u}}1
    {Â}{{\^A}}1 {Ê}{{\^E}}1 {Î}{{\^I}}1 {Ô}{{\^O}}1 {Û}{{\^U}}1
    {œ}{{\oe}}1 {Œ}{{\OE}}1 {æ}{{\ae}}1 {Æ}{{\AE}}1 {ß}{{\ss}}1
    {ç}{{\c c}}1 {Ç}{{\c C}}1 {ø}{{\o}}1 {å}{{\r a}}1 {Å}{{\r A}}1
    {€}{{\EUR}}1 {£}{{\pounds}}1 {^}{\text{\^{}}}1 {\\}{{$\textbackslash$}}1
    {\%}{{\%}}1 
}

\lstdefinestyle{javaBruno}{
  inputencoding=utf8,
%   extendedchars=false,
    language=java,
    backgroundcolor=\color{white},
    commentstyle=\color{dkgreen},
    % keywordstyle=\color{blue},
    % keywordstyle={[2]\color{magenta}},
    numberstyle=\tiny\color{gray},
    stringstyle=\color{mauve},
    % basicstyle=\footnotesize,
    basicstyle=\tiny,
    comment=[l]{\#},
    escapechar=@,
    escapeinside={\%*}{\text{\%}*)},
    % escapeinside={;@}{\^^M},
    otherkeywords={*,...},
    % escapeinside={\%*}{*)},
    % keywords={@relation,@attribute,@data},
    % morekeywords=[2]{real,integer,numeric,string,date},
    breakatwhitespace=true,
    breaklines=true,
    captionpos=b,
    keepspaces=true,
    firstnumber=1,
    numbers=left,
    numbersep=5pt,
    showspaces=false,
    showstringspaces=false,
    showtabs=false,
    tabsize=2,
    frame=single,
    rulecolor=\color{black},
    keywordstyle=\color{blue},
    % morekeywords={*,...},
    % alsoletter={., [\_]},
    texcl=true,
    alsoletter ={_},
    otherkeywords = {!,!=,~,$,*,\&,\%/\%,\%*\%,\%\%,<-,<<-},
    % morecomment=[s][\color{black}]{<!--}{-->},
    stepnumber=1,
    % morecomment=[l][]{//}, 
    % morecomment=[s][]{/*}{*/},
    % morestring=[b]",
    % morestring=[b]',
    extendedchars=true,
    literate=*
    {á}{{\'a}}1 {é}{{\'e}}1 {í}{{\'i}}1 {ó}{{\'o}}1 {ú}{{\'u}}1
    {Á}{{\'A}}1 {É}{{\'E}}1 {Í}{{\'I}}1 {Ó}{{\'O}}1 {Ú}{{\'U}}1
    {à}{{\`a}}1 {è}{{\`e}}1 {ì}{{\`i}}1 {ò}{{\`o}}1 {ù}{{\`u}}1
    {À}{{\`A}}1 {È}{{\'E}}1 {Ì}{{\`I}}1 {Ò}{{\`O}}1 {Ù}{{\`U}}1
    {ä}{{\"a}}1 {ë}{{\"e}}1 {ï}{{\"i}}1 {ö}{{\"o}}1 {ü}{{\"u}}1
    {Ä}{{\"A}}1 {Ë}{{\"E}}1 {Ï}{{\"I}}1 {Ö}{{\"O}}1 {Ü}{{\"U}}1
    {â}{{\^a}}1 {ê}{{\^e}}1 {î}{{\^i}}1 {ô}{{\^o}}1 {û}{{\^u}}1
    {ã}{{\~a}}1 {ẽ}{{\~e}}1 {ĩ}{{\~i}}1 {õ}{{\~o}}1 {ũ}{{\~u}}1
    {Â}{{\^A}}1 {Ê}{{\^E}}1 {Î}{{\^I}}1 {Ô}{{\^O}}1 {Û}{{\^U}}1
    {œ}{{\oe}}1 {Œ}{{\OE}}1 {æ}{{\ae}}1 {Æ}{{\AE}}1 {ß}{{\ss}}1
    {ç}{{\c c}}1 {Ç}{{\c C}}1 {ø}{{\o}}1 {å}{{\r a}}1 {Å}{{\r A}}1
    {€}{{\EUR}}1 {£}{{\pounds}}1 {^}{\text{\^{}}}1 {\\}{{$\textbackslash$}}1
    {\%}{{\%}}1 
}

\lstset{style=javaBruno}




\definecolor{cinColor}{HTML}{008f4c} %cor padrão

\definecolor{consulta}{HTML}{f0f0f5} %cor padrão
\definecolor{consultatitulo}{HTML}{9595b7} %cor padrão

\makeatletter%
\@ifclassloaded{beamer}%
  {\setbeamercolor{saibamais}{fg=cinzaescuro,bg=cinzaclaro}
  	\setbeamercolor{saibamaistitulo}{fg=aliceblue,bg=cinzaescuro}
  	\renewcommand{\inputFilesMaxValue}{20}
  	\renewcommand{\inputCurrentLesson}{-1}
  }{}%
\makeatother%


\titulo{Organização de Computadores}
\disciplina{J554 - ORGANIZAÇÃO DE COMPUTADORES}
\turma{DS4P06-3706}
\autor{Prof. Dr. Bruno Iran Ferreira Maciel}
\cargahoraria{60h}
\alocacao{xx}
\local{OLINDA}
\data{\mydate\today}

\instituicao{}%Faculdade De Informática Do Recife}
\nomecurso{Curso Superior de Tecnologia em Análise e Desenvolvimento de Sistemas}
\programa{}%Graduação em Sistemas de Informação}
% \emailprograma{posgraduacao@cin.ufpe.br}
% \siteprograma{http://cin.ufpe.br/\textasciitilde posgraduacao}

\siteprograma{\href{http://brunomaciel.com}{\beamerbutton{brunomaciel.com}}}%FACIR}

\figuralogo{imagens/logo.png}
\figuralogocapa{imagens/logo-capa.png}
\figuradmca{imagens/fig-dmca.png}



\pgfplotstableread[col sep=&,header=true]{
N & Aulas & Mês & Data & Conteúdo Previsto
1 & 1-3 & Fevereiro & 18/02/2022 & Apresentação da disciplina e boas vindas
2 & 4-6 & Fevereiro & 25/02/2022 & {Máquinas multiníveis contemporâneas}
3 & 7-9 & Março & 04/03/2022 & {Evolução das máquinas multiníveis}
4 & 10-12 & Março & 11/03/2022 & {Processadores, Memória principal e secundária}
5 & 13-15 & Março & 18/03/2022 & {Dispositivos de Entrada e saída}
6 & 16-18 & Março & 25/03/2022 & Aula de Revisão
7 & 19-21 & Abril & 01/04/2022 & Primeira Avaliação - NP1
8 & 22-24 & Abril & 08/04/2022 & {Introdução à portas lógicas}
9 & 25-27 & Abril & 15/04/2022 & Feriado
10 & 28-30 & Abril & 22/04/2022 & {Clocks, RAMs, ROMs, Chips de memória e CPU }
11 & 31-33 & Abril & 29/04/2022 & {Barramentos, Interfaceamento de E/S}
12 & 34-36 & Maio & 06/05/2022 & {Nível de microarquitetura (microprogramação)}
13 & 37-39 & Maio & 13/05/2022 & Aula de Revisão
14 & 40-42 & Maio & 20/05/2022 & Segunda Avaliação - NP2
15 & 43-45 & Maio & 27/05/2022 & Atividades extracurriculares
16 & 46-48 & Junho & 03/06/2022 & Prova substitutiva
17 & 49-51 & Junho & 10/06/2022 & Atividades extracurriculares
18 & 52-54 & Junho & 17/06/2022 & Atividades extracurriculares
19 & 55-57 & Junho & 24/06/2022 & Atividades extracurriculares
20 & 58-60 & Junho & 27/06/2022 & Atividades extracurriculares
}\cronograma

\newcommand{\columnIndex}{4}


% \doublespace
\renewcommand{\baselinestretch}{1.5}

%=========================================================
%------------DIGITE AQUI
%=========================================================
\providecommand{\prova}[1]{\renewcommand{\prova}{#1}}
\providecommand{\semestre}[1]{\renewcommand{\semestre}{#1}}
\providecommand{\ano}[1]{\renewcommand{\ano}{#1}}
\providecommand{\professor}[1]{\renewcommand{\professor}{#1}}
\providecommand{\datava}[1]{\renewcommand{\datava}{#1}}
\prova{Exercício} 
\semestre{1} 
\ano{2020} 
\professor{\autor}
\datava{28/05/2020}

% \newcommand{\orgao}{Faculdade de Informática do Recife (FACIR)}
% \newcommand{\escola}{Análise e Desenvolvimento de Sistemas}
% \newcommand{\disciplina}{288s - Linguagens e Técnicas de Programação}
% \newcommand{\professor}{Bruno Iran Ferreira Maciel}
% \newcommand{\data}{21 de Novembro de 2019}%\today}

% \newcommand{\tempo}{120 Minutos}
% % \newcommand{\aluno}{\bf Aluno:}
% \newcommand{\nota}{NOTA:}
%=========================================================
%------------Cabeçalho da Segunda 
%=========================================================
\pagestyle{headandfoot}%{head}%empty
\firstpageheader{}{}{}
\runningheader{\prova}{}{\data}
\runningheadrule
\firstpagefooter{}{}{}
\runningfooter{}{Bons estudos!}{Pag. \thepage\ de \numpages}
\runningfootrule
%========================================================


% \newcommand{\answerbox}[1][3\baselineskip]{%
%     \noindent\framebox[\linewidth]{%
%         \raisebox{0pt}[0pt][#1]{}%
%     }\par\medskip%
% }



\renewcommand{\solutiontitle}{\noindent\textbf{Resposta:}\par\noindent}


\begin{document}




\vspace{-2cm}
\begin{center}
\resizebox{.95\textwidth}{50mm}{
  \begin{tikzpicture}
    \node[draw=black,shade,
      top color=white!40,
      bottom color=black!5,
      rounded corners=6pt,
      blur shadow={shadow blur steps=5}
    ] {
    \noindent\fontsize{10pt}{10}\selectfont{
    % \begin{minipage}[c]{0.21\linewidth}
    %     % \includegraphics[width=4cm]{outros/logo.png}\\
    %     % \centering
    %     % \textbf{\prova}
    % \end{minipage}
    \begin{minipage}[c]{0.71\linewidth}
        \vspace{5mm}
        {\renewcommand{\arraystretch}{1}
        \begin{tabular}{p{11cm}}
            \instituicao  \textbf{\ano.\semestre} \\
            \multicolumn{1}{l}{\nomecurso} \\
            \multicolumn{1}{l}{\professor} \\
            \multicolumn{1}{l}{DATA DE ENTREGA: \datava} \\
            \multicolumn{1}{l}{\textbf{ALUNO:} \dashsign\dashsign\dashsign} \\
            \multicolumn{1}{l}{\textbf{TURMAS:} \turma  ~~ \textbf{MATRÍCULA:} \dashsign}
        \end{tabular}}
        \vspace{7mm}
    \end{minipage}
    \begin{minipage}[c]{0.23\linewidth}
        \begin{tikzpicture}
            \node[draw=black,shade,
              top color=white!40,
              bottom color=black!5,
              rounded corners=6pt,
              blur shadow={shadow blur steps=5}
            ] {
            \parbox{3cm}{\vspace{13mm}\hspace{2.7cm}}
            };
        \end{tikzpicture}\\
        \centering
        \disciplina
    \end{minipage}
    }
    };
  \end{tikzpicture}}
\end{center}

    
  
  
% \begin{adjustbox}{width=0.5\textwidth,center}





\fbox{\parbox{17.7cm}{
\centering
\fontsize{11pt}{14}\selectfont{\textbf{INFORMAÇÕES IMPORTANTES}}

\fontsize{10pt}{11pt}\selectfont{
\begin{itemize}
    \setlength\itemsep{-0.5em}
    % \item A atividade pode ser encontrada no link \url{https://www.dropbox.com/s/h9a3tt6dnru69j2/jn_2020_1_logica_programacao_1a-va.pdf?dl=0};
    
    % \item Em caso de dúvidas sobre as informações solicitadas, por favor verificar no Portal Acadêmico \url{http://aluno.sereduc.com};

    \item A interpretação da atividade é parte integrante da avaliação;

    \item Em caso dificuldades, entrar em contato comigo por e-mail facir@brunomaciel.com;

    \item Não se esqueça de preencher o número da matrícula, nome complementar e turma;

    \item A avaliação deverá ser entregue até a data da prova;

    \item Qualquer atividade entregue após a data da prova, será penalizada com 20\% a menos. Prazo máximo para envio com atraso será 1 dia após a data da prova e deve ser feito por e-mail - meu e-mail já foi informado aqui;

    \item Será atribuída nota zero ao aluno que entregar sua avaliação em branco ou ter feito plágio de qualquer questão;
    
    \item As questões podem ser feitas usando papel e caneta, mas devem ser entreguem em formato digital com extensão PDF;

    \item Salve o arquivo com nome utilizando seu número de matrícula e nome completo, substituam os espaços em branco por traço (-) e não usem acentuação, exemplo de máscara <matrícula-primeiro-completo>.extensão. Exemplos: ``23958474-bruno-iran-ferreira-maciel.pdf'', ``5384698-marcos-da-silva-junior.pdf'';

    \item Faça o envio de sua atividade preenchendo com seus dados e enviando o arquivo pelo formulário indicado pelo professor. %\url{https://forms.gle/f3Z34P3uT1r4DJHZA};
    
    % \item Após fazer envio do arquivo pelo formulário confirme aqui neste formulário \url{https://forms.office.com/Pages/ResponsePage.aspx?id=u5fvOq0_HUu52FdOJtGShJAACcZmH-hDgv3O4_GIdIxUQzNQNUpQVTU2NkhKU0RVSjNNVUlFV08zVi4u} seu envio. A confirmação é necessária para registro de sua participação na plataforma MS Teams.


\end{itemize}
}}}

\fontsize{12}{12}\selectfont

%--------------------Linha Após a Tabela
\vspace{-10mm}


\newpage


\begin{center}
% \rule[1ex]{\textwidth}{1pt}
{\LARGE{\disciplina}}\\
% Duração: \tempo \hspace{9cm} \\ %Data:\hspace{1cm}26/09/2019\\
% \rule[2ex]{\textwidth}{1pt}
\end{center}


 

\begin{questions}
%=========================================================
%-------------------QUESTÕES DA PROVA
%=========================================================


\question Geral. Defina o conceito de Banco de Dados.
\question Geral. Define o conceito de Sistema de Gerenciamento de Banco de Dados (SGBD).
\question Geral. Qual a utilidade de um SGBD? Apresente as principais características de um SGBD.
\question Geral. Quais as vantagens de um SGBD?
\question Modelo de Dados. Explica os Modelo de Dados:
\begin{itemize}
    \item alto nível;
    \item baixo nível;
    \item presentação de dados.
\end{itemize}
\question Modelo Entidade e Relacionamento (MER). Defina o conceito de modelo de entidade e relacionamento.
\question Modelo Entidade e Relacionamento (MER). Na notação de Peter Chen, qual símbolo representa: entidade, relacionamento e atributo.
\question Modelo Entidade e Relacionamento (MER). Defina os conceitos de:
\begin{itemize}
    \item esquema;
    \item instância;
    \item entidade;
    \item relacionamento;
    \item atributo.
\end{itemize}

\question Arquitetura de SBGD. Explique a arquitetura de três esquemas e descreva cada um dos níveis de esquemas.
\question Arquitetura de SBGD. Defina independência de dados e explique independência lógica de dados.
\question Arquitetura de SBGD. Qual a linguagem padrão para manipulação de dados em SGBD?
\question Arquitetura de SBGD. Defina o conceito e explique ODBC.

\question Modelo Entidade e Relacionamento (MER). Utilize o programa brModelo para construir o modelo conceito do exemplo dado a seguir. Aluno(matrícula, cpf, nome, idade, peso, curso, período), curso(nome), faculdade(cpnj, nome, logradouro, curso). Considere definir o relacionamento entre aluno e curso, assim como faculdade e curso.

\question  Modelo Entidade e Relacionamento (MER). Utilize o programa brModelo para construir o modelo lógico do exemplo dado a seguir. Aluno(matrícula, cpf, nome, idade, peso, curso, período), curso(nome), faculdade(cpnj, nome, logradouro, curso).
\question Modelo Entidade e Relacionamento (MER). Explique a diferença entre MER e DER.
\question Modelo Entidade e Relacionamento (MER). Defina os conceitos:
\begin{itemize}
    \item grau de relação;
    \item cardinalidade;
    \item domínio;
    \item super chave;
    \item chave;
    \item chave primária;
    \item chave candidata;
    \item chave alternativa;
    \item chave estrangeira;
    \item entidade fraca;
    \item entidade forte;
    \item atributo multivalorado.
\end{itemize}

\question Explique o conceito de  em Modelo Relacional (MR).
\question Explique os conceitos de restrições em Modelo Relacional (MR).
\begin{itemize}
    \item atributo NULO;
    \item chave;
    \item domínio.
\end{itemize}

\question Modelo Relacional (MR). Explique as regras de \textbf{integridade de entidade} e \textbf{integridade referencial}.
\question Modelo Relacional (MR). Explique o que é e pra serve: mapeamento entre MER para MR.

\question Álgebra Relacional. Defina o conceito de álgebra relacional.
\question Álgebra Relacional. Explique as operações:
\begin{itemize}
    \item PROJEÇÃO
    \item ATRIBUIÇÃO
    \item RENOMEAÇÃO
    \item UNIÃO
    \item INTERSECÇÃO
    \item SUBTRAÇÃO
    \item PRODUTO CARTESIANO
    \item JUNÇÃO
\end{itemize}


\question Normalização. Explique o conceito de Normalização.
\question Normalização. Explique as diferenças entre as técnicas BOTTOM-UP e TOP-DOWN.
\question Normalização. Explique o conceito de Dependência Funcional:
\begin{itemize}
    \item separada;
    \item acumulada;
    \item transitiva.
\end{itemize}
\question Normalização. Explique o que ocorre nas formas normais:
\begin{itemize}
    \item 1FN;
    \item 2FN;
    \item 3FN.
\end{itemize}

\question Administração de SGBD. Explique quais os princípios de administra de SGBD.
\question Administração de SGBD. Explique os conceitos:
\begin{itemize}
    \item concorrência;
    \item backup;
    \item restauração;
    \item bloqueio: compartilhado e exclusivo;
    \item segurança;
    \item consulta.
\end{itemize}
\question Fundamentos de SQL. Explique o conceito de SQL.
% \question Fundamentos de SQL. Defina a instrução para:
% \begin{itemize}
%     \item criar um esquema com nome facir;
%     \item criar tabelas com nomes empregado(nome varchar(20), cpnj varchar(15)), aluno(cpf varchar(11), nome (varchar(20)) e curso(codigo int, nome varchar(20)).
% \end{itemize}

\question Bancos de dados são conjuntos de arquivos que se comunicam entre si, armazenando uma vasta gama de dados, tais como nomes, documentos, pagamentos, endereços, clientes, etc. São configurados e gerenciados por meio das linguagens de programação, como Javascript, SQL, PL/SQL, entre outras. Referente a banco de dados, em SQL, o comando utilizado para criar um esquema é

\begin{parts}
    \part CREATE TABLE AUTHORIZATION SQL;
    \part CREATE SCHEMA;
    \part CREATE TABLE SCHEMA;
    \part DEFINE SCHEMA;
    \part SCHEMA CREATE;
\end{parts}

\question Dada a Linguagem de Definição de Dados (DDL) em Bancos de dados, assinale a alternativa INCORRETA.

\begin{parts}
    \part Permite definir tabelas;
    \part Permite definir restrições de integridade;
    \part Permite realizar um comando ``UPDATE'';
    \part Permite definir assertivas;
    \part Permite alterar o tipo de dado de uma determinada coluna. 
\end{parts}

\question No que diz respeito a banco de dados, julgue. Na linguagem SQL (Structured Query Language), os comandos CREATE, ALTER e DROP fazem parte da linguagem de manipulação de dados (DML). 

\begin{parts}
    \part Sim;
    \part Não. 
\end{parts}



\question Considere o comando SQL abaixo.

\begin{lstlisting}[language=sql]
CREATE TABLE FACULDADE(
    cnpj varchar(15) PRIMARY KEY,
    nome varchar(20),
    logradouro varchar(50)
);
\end{lstlisting}
\vspace{1mm}
Um Estagiário utilizou outro comando SQL, sem erros de sintaxe e com valores válidos, para inserir dados na tabela. O comando correto por ele utilizado foi: 

\begin{parts}
    \part INSERT INTO FACULDADE VALUES ('023698745896325', 'FACIR', 'RUA DO SOL');
    \part INSERT INTO FACULDADE ('023698745896325', 'FACIR', 'RUA DO SOL');
    \part INSERT IN FACULDADE VALUES ('023698745896325', 'FACIR', 'RUA DO SOL');
    \part INSERT INTO FACULDADE VALUES ('023698745896325', 'FACIR', RUA DO SOL);
    \part INSERT IN FACULDADE ('023698745896325', 'FACIR', 'RUA DO SOL'); 
\end{parts}


\question Considere o comando SQL abaixo.

\begin{lstlisting}[language=sql]
CREATE TABLE FACULDADE(
    cnpj varchar(15) PRIMARY KEY,
    nome varchar(20),
    logradouro varchar(50)
);
\end{lstlisting}
\vspace{1mm}
Considerando que já existem muitos registros na tabela, para selecionar aqueles cujo nomes sejam iniciados por 'FACIR', deve-se utilizar o comando SQL.

\begin{parts}
    \part SELECT ALL RECORDS FROM FACULDADE WHERE nome > 'FACIR';
    \part SELECT * FROM FACULDADE WHERE nome LIKE 'FACIR\%';
    \part SELECT ALL FROM FACULDADE WHERE nome LIKE '\%FACIR\%';
    \part SELECT *.* FROM FACULDADE WHERE nome = 'FACIR'
    \part SELECT FROM FACULDADE ALL RECORDS WHERE nome IN 'FACIR'; 
\end{parts}


% \makeemptybox{1in}

% \question[1] As características que definem o pensamento computacional são ?

% \begin{parts}
%     \part reconhecimento de padrões, ilusão, remoção de detalhes e algoritmos.
%     \part abstração, resultado, algoritmos e divisão.
%     \part decomposição, reconhecimento de padrões, abstração e algoritmos.
%     \part generalização, divisão, abstração e algoritmos.
%     \part N.D.A
% \end{parts}


% \question[1] O pensamento computacional baseia-se em quatro pilares que orientam o processo de solução de problemas. Em relação ao algoritmo é CORRETO afirmar:

% \begin{parts}
%     \part engloba todos os pilares e é o processo de criação de um conjunto de regras para a resolução do problema.
%     \part sempre será a parte codificada da solução.
%     \part responsável pelo reconhecimento de padrões.
%     \part escrita do código em linguagem de programação.
%     \part N.D.A
% \end{parts}


% \question[2] Sistema de caixa de loja. Ler nome do produto, quantidade e preço unitário. Calcular e escrever o valor total (VT), o valor do desconto e o valor total a pagar (VTP).

% \begin{itemize}
%     \item valor total (VT) = quantidade * preço unitário
%     \item valor total a pagar (VTP) = valor total - valor desconto
% \end{itemize}


% sabendo-se que o valor do desconto é calculado da seguinte maneira.

% para produtos com preço unitário \textbf{menor que} R\$ 100:

% \begin{itemize}
%     \item Se quantidade <= 2 o desconto será de 1\%
%     \item Se quantidade > 2 e quantidade <=15 o desconto será de 4\%
%     \item Se quantidade > 15 o desconto será de 5\%
% \end{itemize}

% para produtos com preço unitário \textbf{maior ou igual que} R\$ 100:

% \begin{itemize}
%     \item Se quantidade <= 2 o desconto será de 3\%
%     \item Se quantidade > 2 e quantidade <=15 o desconto será de 5\%
%     \item Se quantidade > 15 o desconto será de 7\%
% \end{itemize}

% \question[2] Sistema de calculadora. Ler 4 valores inteiros. Se qualquer valor informado for menor ou igual a ZERO, deve ser lido um novo valor, ou seja, nenhum valor aceito pode ser menor ou igual a ZERO. Calcule e escreva ao final o resultado da soma dos dois maiores números lidos.

% \question[1] Sistema de ordenação de valores. Ler 5 valores (considere que não serão informados valores iguais). Escrever os números em ordem CRESCENTE.

% \question[1] Sistema de ordenação de valores. Ler 5 valores (considere que não serão informados valores iguais). Escrever os números em ordem DECRESCENTE.




%=========================================================
%-------------------FIM DA PROVA
%=========================================================
\end{questions}


\end{document}