\vspace{-1cm}
\begin{center}
  \begin{tikzpicture}
    \node[draw=black,shade,
      top color=white!40,
      bottom color=black!5,
      rounded corners=6pt,
      blur shadow={shadow blur steps=5}
    ] {
    
    \noindent\fontsize{8pt}{10}\selectfont{
    % \begin{minipage}[c][6cm][t]{1\linewidth}
    \begin{minipage}[c]{0.21\linewidth}
        \includegraphics[width=4cm]{imagens/logo.png}\\
        \centering
        \textbf{\prova}
    \end{minipage} % no space if you would like to put them side by side
    \begin{minipage}[c]{0.51\linewidth}
        \vspace{10mm}
        \begin{tabular}{ll}
            \instituicao & \textbf{\ano.\semestre} \\
            \multicolumn{2}{l}{\nomecurso} \\
            \professor & DATA: \datava  \\
            \multicolumn{2}{l}{\textbf{ALUNO:} \dashsign\dashsign\dashsign} \\
            \multicolumn{2}{l}{\textbf{TURMA:} \dashsign ~~ \textbf{MATRÍCULA:} \dashsign}
        \end{tabular}
        \vspace{7mm}
    \end{minipage}
    \begin{minipage}[c]{0.20\linewidth}
        \begin{tikzpicture}
            \node[draw=black,shade,
              top color=white!40,
              bottom color=black!5,
              rounded corners=6pt,
              blur shadow={shadow blur steps=5}
            ] {
            \parbox{3cm}{\vspace{13mm}\hspace{2.7cm}}
            };
        \end{tikzpicture}\\
        \centering
        \disciplina
    \end{minipage}
    % \end{minipage}
    }
    };
  \end{tikzpicture}
\end{center}

    
  
  
% \begin{adjustbox}{width=0.5\textwidth,center}





\fbox{\parbox{17.7cm}{
\centering
\fontsize{11pt}{14}\selectfont{\textbf{INFORMAÇÕES IMPORTANTES:}}

\fontsize{10pt}{11}\selectfont{
\begin{itemize}
    \setlength\itemsep{-0.3em}
    \item A interpretação da atividade é parte integrante da avaliação; 
    % \item Não é permitido o uso de telefones celulares nem de equipamentos eletrônicos;
    \item Faça o envio pelo formulário disponibilizado no Google Sala de Aula. Em caso dificuldades, entrar em contato comigo por e-mail esuda@brunomaciel.com;
    \item Não se esqueça de preencher o número da matrícula e nome complementar dentro do documento;
    % \item Ao final da avaliação não se esqueça de assinar a ata de avaliação;
    % \item Não é permitida consulta de qualquer natureza, EXCETO quando autorizada pelo professor titular da disciplina;
    % \item Escolha uma questão e disserte valendo de (0) zero a (10) dez pontos com no mínimo 1 lauda.
    \item A avaliação deverá ser respondida até a data 13/04/2020 no horário antes da aula, ou seja, deve ser entregue antes das 18h;
    \item Qualquer atividade entregue após as 18h, será penalizada com 20\% a menos. Prazo máximo para envio com atraso será 14/04/2020;
    \item Escreva o máximo que puder para obter a nota máxima;
    \item Detalhe os conceitos e explicações para que seja possível que eu compreenda seu aprendizado;
    \item Salve o arquivo com nome utilizando seu número de matrícula e nome completo, substituam os espaços em branco por traço (-) e não usem acentuação, exemplo de máscara <matrícula-primeiro-completo>.extensão. Exemplos: "23958474-bruno-iran-ferreira-maciel.pdf", "5384698-marcos-da-silva-junior.doc";
    \item Será atribuída nota zero ao aluno que devolver sua avaliação em branco ou ter feito plágio de qualquer questão.
    % \item As questões de 1 à 18 valem 10 pontos;
    % \item A questão 19, vale 5 pontos, MAS está condicionada - se optar por fazer a questão 19, escolha 9 questões e entregue APENAS as respostas para as 9 questões.
\end{itemize}
}}}



% \end{adjustbox}

% \begin{center}
% \begin{tabular}{|l|l|l|l|l|l|}
%  \hline
%     \multicolumn{6}{|c|}{\cellcolor{black!20}Folha de Resposta} \\ \hline 
%     \multicolumn{6}{|c|}{GABARITO} \\ \hline 
    
%     \cellcolor{black!20}\textbf{1} & A & B & C & D & E \\ \hline \cellcolor{black!20}\textbf{2} & A & B & C & D & E \\ \hline \cellcolor{black!20}\textbf{3} & A & B & C & D & E \\ \hline \cellcolor{black!20}\textbf{4} & \multicolumn{5}{|c|}{Dissertativa}  \\ \hline
    
%     \multicolumn{6}{|c|}{} \\ \hline
% \end{tabular}
% \end{center}


% \begin{center}
% \begin{tabular}{|l|l|l|l|l|l|}
%  \hline
%     \multicolumn{6}{|c|}{\cellcolor{black!20}Folha de Resposta} \\ \hline 
%     \multicolumn{6}{|c|}{GABARITO} \\ \hline 
    
%     \cellcolor{black!20}\textbf{1} & A & B & C & \cellcolor{black!50}D & E \\ \hline \cellcolor{black!20}\textbf{2} & A & B & C & D & \cellcolor{black!50}E \\ \hline \cellcolor{black!20}\textbf{3} & A & B & \cellcolor{black!50}C & D & E \\ \hline
%     \cellcolor{black!20}\textbf{4} & \multicolumn{5}{|c|}{Dissertativa}  \\ \hline
    
%     \multicolumn{6}{|c|}{} \\ \hline
% \end{tabular}
% \end{center}



% \begin{center}
% \fbox{\parbox{15cm}{
% \textbf{Observação}: na Folha de Resposta, a marcação das letras correspondentes às respostas assinaladas por você para as questões de múltipla escolha (apenas uma resposta por questão) deve ser feita cobrindo a letra e preenchendo todo o espaço compreendido pelo retângulo que a envolve, de forma contínua e densa, com caneta esferográfica preta ou azul.\\
% \textbf{A Rasura anulará a questão.}
% }}
% \end{center}

% \begin{center}
% \begin{tabular}{|l|l|l|l|l|l|}
%  \hline
%     Exemplo & A & \cellcolor{black}B & C & D & E \\ \hline
% \end{tabular}
% \end{center}